

% --- Ngon ngu & Font ---
\usepackage[utf8]{inputenc}       % Ma hoa (encoding) de go duoc Tieng Viet
\usepackage[T5]{fontenc}          % Font encoding cho Tieng Viet
\usepackage[vietnamese]{babel}    % Thiet lap cac quy tac Tieng Viet (vd: Chuong 1)

% --- Toan hoc ---
\usepackage{amsmath}              % Cac cong thuc toan hoc
\usepackage{amssymb}   
\usepackage{amsthm}
\newtheorem{theorem}{Định lý}[chapter]
\newtheorem{warning}{Cảnh báo}[section]
\newtheorem{example}{Ví dụ}[section]           % Cac ky hieu toan hoc

% --- Hinh anh & Mau sac ---
\usepackage{graphicx}             % De chen hinh anh (\includegraphics)
\usepackage{xcolor}               % De su dung mau sac

% --- Layout & Hyperlink ---
\usepackage{geometry}             % De can chinh le trang
\usepackage[
    unicode=true, 
    colorlinks=true,
    linkcolor=blue,
    urlcolor=blue,
    citecolor=red
]{hyperref}                   % De tao hyperlink trong tai lieu

% --- Thiet lap le trang (Tuy chon) ---
\geometry{a4paper, top=2.5cm, bottom=2.5cm, left=2.5cm, right=2.5cm}

% --- Cac lenh tuy chinh khac ---
% (Ban co the them cac \newtheorem, \newcommand... vao day)